\documentclass[]{article}
\usepackage{lmodern}
\usepackage{amssymb,amsmath}
\usepackage{ifxetex,ifluatex}
\usepackage{fixltx2e} % provides \textsubscript
\ifnum 0\ifxetex 1\fi\ifluatex 1\fi=0 % if pdftex
  \usepackage[T1]{fontenc}
  \usepackage[utf8]{inputenc}
\else % if luatex or xelatex
  \ifxetex
    \usepackage{mathspec}
  \else
    \usepackage{fontspec}
  \fi
  \defaultfontfeatures{Ligatures=TeX,Scale=MatchLowercase}
\fi
% use upquote if available, for straight quotes in verbatim environments
\IfFileExists{upquote.sty}{\usepackage{upquote}}{}
% use microtype if available
\IfFileExists{microtype.sty}{%
\usepackage{microtype}
\UseMicrotypeSet[protrusion]{basicmath} % disable protrusion for tt fonts
}{}
\usepackage[margin=1in]{geometry}
\usepackage{hyperref}
\hypersetup{unicode=true,
            pdftitle={Veiledning},
            pdfauthor={Intervjukomiteen},
            pdfborder={0 0 0},
            breaklinks=true}
\urlstyle{same}  % don't use monospace font for urls
\usepackage{color}
\usepackage{fancyvrb}
\newcommand{\VerbBar}{|}
\newcommand{\VERB}{\Verb[commandchars=\\\{\}]}
\DefineVerbatimEnvironment{Highlighting}{Verbatim}{commandchars=\\\{\}}
% Add ',fontsize=\small' for more characters per line
\usepackage{framed}
\definecolor{shadecolor}{RGB}{248,248,248}
\newenvironment{Shaded}{\begin{snugshade}}{\end{snugshade}}
\newcommand{\KeywordTok}[1]{\textcolor[rgb]{0.13,0.29,0.53}{\textbf{#1}}}
\newcommand{\DataTypeTok}[1]{\textcolor[rgb]{0.13,0.29,0.53}{#1}}
\newcommand{\DecValTok}[1]{\textcolor[rgb]{0.00,0.00,0.81}{#1}}
\newcommand{\BaseNTok}[1]{\textcolor[rgb]{0.00,0.00,0.81}{#1}}
\newcommand{\FloatTok}[1]{\textcolor[rgb]{0.00,0.00,0.81}{#1}}
\newcommand{\ConstantTok}[1]{\textcolor[rgb]{0.00,0.00,0.00}{#1}}
\newcommand{\CharTok}[1]{\textcolor[rgb]{0.31,0.60,0.02}{#1}}
\newcommand{\SpecialCharTok}[1]{\textcolor[rgb]{0.00,0.00,0.00}{#1}}
\newcommand{\StringTok}[1]{\textcolor[rgb]{0.31,0.60,0.02}{#1}}
\newcommand{\VerbatimStringTok}[1]{\textcolor[rgb]{0.31,0.60,0.02}{#1}}
\newcommand{\SpecialStringTok}[1]{\textcolor[rgb]{0.31,0.60,0.02}{#1}}
\newcommand{\ImportTok}[1]{#1}
\newcommand{\CommentTok}[1]{\textcolor[rgb]{0.56,0.35,0.01}{\textit{#1}}}
\newcommand{\DocumentationTok}[1]{\textcolor[rgb]{0.56,0.35,0.01}{\textbf{\textit{#1}}}}
\newcommand{\AnnotationTok}[1]{\textcolor[rgb]{0.56,0.35,0.01}{\textbf{\textit{#1}}}}
\newcommand{\CommentVarTok}[1]{\textcolor[rgb]{0.56,0.35,0.01}{\textbf{\textit{#1}}}}
\newcommand{\OtherTok}[1]{\textcolor[rgb]{0.56,0.35,0.01}{#1}}
\newcommand{\FunctionTok}[1]{\textcolor[rgb]{0.00,0.00,0.00}{#1}}
\newcommand{\VariableTok}[1]{\textcolor[rgb]{0.00,0.00,0.00}{#1}}
\newcommand{\ControlFlowTok}[1]{\textcolor[rgb]{0.13,0.29,0.53}{\textbf{#1}}}
\newcommand{\OperatorTok}[1]{\textcolor[rgb]{0.81,0.36,0.00}{\textbf{#1}}}
\newcommand{\BuiltInTok}[1]{#1}
\newcommand{\ExtensionTok}[1]{#1}
\newcommand{\PreprocessorTok}[1]{\textcolor[rgb]{0.56,0.35,0.01}{\textit{#1}}}
\newcommand{\AttributeTok}[1]{\textcolor[rgb]{0.77,0.63,0.00}{#1}}
\newcommand{\RegionMarkerTok}[1]{#1}
\newcommand{\InformationTok}[1]{\textcolor[rgb]{0.56,0.35,0.01}{\textbf{\textit{#1}}}}
\newcommand{\WarningTok}[1]{\textcolor[rgb]{0.56,0.35,0.01}{\textbf{\textit{#1}}}}
\newcommand{\AlertTok}[1]{\textcolor[rgb]{0.94,0.16,0.16}{#1}}
\newcommand{\ErrorTok}[1]{\textcolor[rgb]{0.64,0.00,0.00}{\textbf{#1}}}
\newcommand{\NormalTok}[1]{#1}
\usepackage{graphicx,grffile}
\makeatletter
\def\maxwidth{\ifdim\Gin@nat@width>\linewidth\linewidth\else\Gin@nat@width\fi}
\def\maxheight{\ifdim\Gin@nat@height>\textheight\textheight\else\Gin@nat@height\fi}
\makeatother
% Scale images if necessary, so that they will not overflow the page
% margins by default, and it is still possible to overwrite the defaults
% using explicit options in \includegraphics[width, height, ...]{}
\setkeys{Gin}{width=\maxwidth,height=\maxheight,keepaspectratio}
\IfFileExists{parskip.sty}{%
\usepackage{parskip}
}{% else
\setlength{\parindent}{0pt}
\setlength{\parskip}{6pt plus 2pt minus 1pt}
}
\setlength{\emergencystretch}{3em}  % prevent overfull lines
\providecommand{\tightlist}{%
  \setlength{\itemsep}{0pt}\setlength{\parskip}{0pt}}
\setcounter{secnumdepth}{0}
% Redefines (sub)paragraphs to behave more like sections
\ifx\paragraph\undefined\else
\let\oldparagraph\paragraph
\renewcommand{\paragraph}[1]{\oldparagraph{#1}\mbox{}}
\fi
\ifx\subparagraph\undefined\else
\let\oldsubparagraph\subparagraph
\renewcommand{\subparagraph}[1]{\oldsubparagraph{#1}\mbox{}}
\fi

%%% Use protect on footnotes to avoid problems with footnotes in titles
\let\rmarkdownfootnote\footnote%
\def\footnote{\protect\rmarkdownfootnote}

%%% Change title format to be more compact
\usepackage{titling}

% Create subtitle command for use in maketitle
\providecommand{\subtitle}[1]{
  \posttitle{
    \begin{center}\large#1\end{center}
    }
}

\setlength{\droptitle}{-2em}

  \title{Veiledning}
    \pretitle{\vspace{\droptitle}\centering\huge}
  \posttitle{\par}
    \author{Intervjukomiteen}
    \preauthor{\centering\large\emph}
  \postauthor{\par}
      \predate{\centering\large\emph}
  \postdate{\par}
    \date{\begin{enumerate}
\def\labelenumi{\arabic{enumi}.}
\setcounter{enumi}{17}
\tightlist
\item
  juni 2019
\end{enumerate}}


\begin{document}
\maketitle

\section{Lag en interaktiv resultattjeneste i
R-Shiny}\label{lag-en-interaktiv-resultattjeneste-i-r-shiny}

Beskrivelsen under er ikke nødvendigvis utfyllende og forutsetter noe
kjennskap til RStudio og bruk av git og GitHub. Som en ekstra støtte
anbefales \href{http://r-pkgs.had.co.nz/}{R pacakges} av Hadley Wickham
og spesielt
\href{http://r-pkgs.had.co.nz/git.html\#git-rstudio}{beskrivelsen av git
og GitHub}.

En resultattjeneste er en interaktiv løsning hvor man kan se på
resultater for sine data.Det er utarbeidet et templat (en mal) i Shiny
som man tar utgangspunkt i for å lage en interaktiv resultattjeneste.

\subsection{Prøv templatet}\label{prv-templatet}

\begin{enumerate}
\def\labelenumi{\arabic{enumi}.}
\tightlist
\item
  Installér pakken
  \href{https://github.com/Rapporteket/rapRegTemplate}{rapRegTemplate} i
  RStudio
  (\texttt{devtools::install\_github("Rapporteket/rapRegTemplate")})
\item
  Hent ned prosjektet
  \href{https://github.com/Rapporteket/rapRegTemplate}{rapRegTemplate}
  til RStudio (for mer info, se
  \href{https://support.rstudio.com/hc/en-us/articles/200526207-Using-Projects}{her})
  (Hvis du ikke klarer å sette opp en direkte kobling til github kan du
  kopiere filene til din datamaskin.)
\item
  Åpne fila inst/shinyApps/app1/ui.R og start Shiny-applikasjonen (``Run
  App'')
\item
  Navigér i applikasjonen for å se på struktur og farger (innhold
  mangler)
\end{enumerate}

\subsection{Valgfritt: lag ditt eget prosjekt basert på
templatet}\label{valgfritt-lag-ditt-eget-prosjekt-basert-pa-templatet}

\bf{Are: Skal vi ha med denne? I så fall kan den ikke være valgfri og vi bør kanskje basere oss på datasettet vi benytter til metadataspørsmål.}
Denne delen er satt som valgfri men kan likevel være relevant, særlig om det er ønskelig å benytte templatetet som utgangspunkt for etablering av nye registre på Rapporteket.

\begin{enumerate}
\def\labelenumi{\arabic{enumi}.}
\tightlist
\item
  Lag et nytt prosjekt i RStudio som en R-pakke
\item
  Gi pakken et navn som gjerne gjenspeiler overordnet funksjon i pakken,
  eksempelvis ``testRegister''
\item
  Valgfritt: hak gjerne av for ``Create a git repository'' også da det
  vil gi nyttig kunnskap når egne registre skal utvikles seinere
\item
  Trykk ``Create Project''.
\item
  Kopier inn alle filer fra katalogene ``inst/'' og ``R/'' i
  ``rapRegTemplate'' og legg disse i tilsvarende kataloger i den nye
  pakken
\item
  I toppen av ``server.ui'' endre avhengigheten til R-pakken
  ``rapRegTemplate'' til din egen R-pakke
\item
  Endre DESCRIPTION-fila slik at den blir nogenlunde tilssvarende den
  som finnes i ``rapRegTemplate'', særlig det som er gitt under
  ``Depends:'', ``Imports:'' og ``Remotes:''
\item
  Bygg, installér og last pakken i R
\item
  Test gjerne at innebygget Shiny-applikasjon fungerer på samme vis som
  i prosjektet ``rapRegTemplate''
\end{enumerate}

\subsection{Last data}\label{last-data}

\begin{enumerate}
\def\labelenumi{\arabic{enumi}.}
\tightlist
\item
  Åpne fila R/GetFakeRegData.R
\item
  Se at funksjonen returnerer et kjedelig og irrelevant innebygget
  datasett :-(
\item
  Prøv funksjonen fra kommandolinja (Console i RStudio), \emph{e.g.}
  \texttt{df\ \textless{}-\ getFakeRegData()}
\item
  Sjekk at du får returnert ei dataramme med X observasjoner for Y
  variabler, \emph{e.g.} \texttt{attributes(df)}
\end{enumerate}

\subsection{Lag innhold i Shiny-applikasjonen, steg
1}\label{lag-innhold-i-shiny-applikasjonen-steg-1}

Utgangspunket for de neste stegene er bruk av det innebygde datasettet
``mtcars'', jf. ``Alternativ 2'' over.

\begin{enumerate}
\def\labelenumi{\arabic{enumi}.}
\tightlist
\item
  I shiny-applikasjonen, navigér til arkfanen ``Figur og tabell''
\item
  Åpne fila inst/shinyApps/app1/ui.R
\item
  Bla ned til linja \texttt{tabPanel("Figur\ og\ tabell"}
\item
  Kommenter inn linjene under, lagre fila og last applikasjonen på nytt
  (``Reload App'')
\item
  Sjekk at det er kommet inn GUI-elementer i arkfanen ``Figur og
  tabell'' som før var tom
\item
  Prøv gjerne de brukervalg som er i venstre kolonne
\item
  Oppgave: gjør endringer i inst/shinyApps/app1/ui.R (på de linjene som
  nettopp er kommentert inn) slik at maks antall grupper endres fra 10
  til 12 i applikasjonen
\end{enumerate}

\subsection{Lag innhold i Shiny-applikasjonen, steg
2}\label{lag-innhold-i-shiny-applikasjonen-steg-2}

\begin{enumerate}
\def\labelenumi{\arabic{enumi}.}
\tightlist
\item
  Åpne fila inst/shinyApps/app1/server.R
\item
  Bla ned til kommentaren \texttt{\#\ Last\ inn\ data} og kommenter inn
  linja under
\item
  Bla videre ned til kommentaren \texttt{\#\ Figur\ og\ tabell} og
  kommenter inn de linjene som ligger under \texttt{\#\#\ Figur} og
  \texttt{\#\#\ Tabell}, hhv
\item
  Sjekk at det er samsvar mellom id-ene definert i
  inst/shinyApps/app1/ui.R og de datastrukturene
  (\texttt{output\$distPlot} og \texttt{output\$distTable}) du nå har
  definert i inst/shinyApps/app1/server.R
\item
  Se at \texttt{regData} gis inn til de funksjoner som lager figur og
  tabell, hhv
\item
  Se også at de samme funksjonene tar i mot de brukervalg som er
  definert i inst/shinyApps/app1/ui.R (\texttt{input\$var} og
  \texttt{input\$bins})
\item
  Valgfritt: ta en titt på funksjonen som lager innholdet i figur og
  tabell: \texttt{?makeHist}
\item
  Lagre fila, start applikasjonen på nytt og sjekk at figur og tabell er
  på plass og at disse reagerer på ulike brukervalg
\item
  Oppgave A: lag en ny arkfane ``Sammendrag'' (etter ``Figur'' og
  ``Tabell'') ved å legge til kode i inst/shinyApps/app1/ui.R
\item
  Oppgave B: fyll ``Sammendrag'' med en tabell som viser
  \texttt{summary} av valgt variabel ved å legge til kode i
  inst/shinyApps/app1/server.R
\end{enumerate}

Tips til oppgave B:

\begin{Shaded}
\begin{Highlighting}[]
\NormalTok{## Sammendrag}
\NormalTok{output}\OperatorTok{$}\NormalTok{distSummary <-}\StringTok{ }\KeywordTok{renderTable}\NormalTok{(\{}
  \KeywordTok{as.data.frame}\NormalTok{(}\KeywordTok{sapply}\NormalTok{(regData, summary))[input}\OperatorTok{$}\NormalTok{var]}
\NormalTok{\}, }\DataTypeTok{rownames =} \OtherTok{TRUE}\NormalTok{)}
\end{Highlighting}
\end{Shaded}

\subsection{Lag innhold i Shiny-applikasjonen, steg
3}\label{lag-innhold-i-shiny-applikasjonen-steg-3}

Bruk samme tilnærming som over, men for ``Samlerapport''.
\bf{Are: Her kan vi be de lage presentasjonen for metadata}
Her er det en del nye elementer, bl.a.

\begin{itemize}
\tightlist
\item
  bruk av en Rmd-fil som rapportmal
\item
  funksjonalitet for nedlasting av rapporten
\end{itemize}

\subsection{Lag innhold i Shiny-applikasjonen, steg
4}\label{lag-innhold-i-shiny-applikasjonen-steg-4}

Denne delen forutsetter bruk av
\href{https://github.com/Rapporteket/docker}{Docker for Rapporteket}
eller tilsvarende utviklingsmiljø som ``simulerer'' Rapporteket. Her
skal hver enkelt bruker kunne bestille rutinemessig tilsending per epost
av gitte rapporter, eksempelvis slik som samlerapporten over med
predefinerte verdier for ``Variabel'' og ``Antall grupper''.
Tilnærmingen introduserer noen nye elementer, slik som:

\begin{itemize}
\tightlist
\item
  reaktive verdier
\item
  lagring av innstillinger som er ``varige'' også etter at appliksjonen
  er avsluttet
\end{itemize}

\subsection{Valgfritt: sjekk inn endringer i
git}\label{valgfritt-sjekk-inn-endringer-i-git}

Git er et verktøy for versjonskontroll som gir mulighet for å spore
endringer og samarbeide om kode. Basale funksjoner i git er svært
nyttinge, men kan virke forvirrende i starten. Sørg for at egen kode
(bestandig) versjonshåndteres (i git) og at koden finnes sentralisert og
tilgjengelig for deg selv og andre (på GitHub).

\begin{enumerate}
\def\labelenumi{\arabic{enumi}.}
\tightlist
\item
  Sett opp git lokalt og etabler et sentralt repository for din R-pakke
  gjennom å følge
  \href{http://r-pkgs.had.co.nz/git.html\#git-rstudio}{Hadley Wickhams
  veiledning}
\item
  Om du ikke har det fra før, etabler et ssh-nøkkelpar for sikker
  kommunikasjon med GitHub
\end{enumerate}

NB Ved etablering av et nøkkelpar for bruk av Secure Shell (ssh) i
kommunikasjonen med GitHub (generelt lurt men også nødvendig for
avansert bruk av Rapporteket) er det viktig å påse at disse blir
etablert på din egen fysiske datamaskin (og eksempelvi ikke inne i en
docker-container om det er i bruk)

\subsection{Valgfritt: dytt (push) R-pakken til
GitHub}\label{valgfritt-dytt-push-r-pakken-til-github}

\begin{enumerate}
\def\labelenumi{\arabic{enumi}.}
\tightlist
\item
  Om du ikke allerede har gjort det, lag din egen bruker på GitHub (se
  over)
\item
  Om du ikke allerede har gjort det,
  \href{https://help.github.com/en/articles/adding-a-new-ssh-key-to-your-github-account}{legg
  ut den offentlige delen av ditt ssh-nøkkelpar på din github-konto}
\item
  Om du ikke allerede har gjort det, bli medlem av organisasjonen
  Rapporteket på GitHub
\item
  Under din egen side på GitHub, opprett et Repository med navn
  tilsvarende din egen pakke (\emph{e.g.} ``testRegister'')
\item
  I RStudio, push pakken til ditt nye Repository på GitHub
\end{enumerate}


\end{document}
